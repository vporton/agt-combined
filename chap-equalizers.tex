\chapter{Equalizers and co-Equalizers in Certain Categories}

It is a rough draft. Errors are possible.

\fxwarning{Change notation $\prod$ $\rightarrow$ $\prod^{(L)}$.}

\section{Equalizers}

Categories $\cont (\mathcal{C})$ are defined above.

I will denote $W$ the forgetful functor from $\cont
(\mathcal{C})$ to $\mathcal{C}$.

In the definition of the category $\cont (\mathcal{C})$ take
values of $\uparrow$ as principal morphisms. \fxwarning{Wording.}

\begin{lem}
  Let $f : X \rightarrow Y$ be a morphism of the category
  $\cont (\mathcal{C})$ where $\mathcal{C}$ is a concrete
  category (so $W f = \uparrow \varphi$ for a $\mathbf{Rel}$-morphism
  $\varphi$ because $f$ is principal) and $\im \varphi = A \subseteq
  \Ob Y$. Factor it $\varphi = \mathcal{E}^{\Ob Y} \circ u$
  where $u : \Ob X \rightarrow A$ using properties of
  $\mathbf{Set}$. Then $u$ is a morphism of $\cont
  (\mathcal{C})$ (that is a continuous function $X \rightarrow \iota_A Y$).
\end{lem}

\begin{proof}
  $(\mathcal{E}^{\Ob Y})^{- 1} \circ \varphi = (\mathcal{E}^{\Ob Y})^{- 1} \circ \mathcal{E}^{\Ob Y} \circ u$;
  
  $(\mathcal{E}_{\mathcal{C}}^{\Ob Y})^{- 1} \circ \uparrow \varphi
  = (\mathcal{E}_{\mathcal{C}}^{\Ob Y})^{- 1} \circ \mathcal{E}_{\mathcal{C}}^{\Ob Y} \circ \uparrow u$;
  
  $(\mathcal{E}_{\mathcal{C}}^{\Ob Y})^{- 1} \circ \uparrow \varphi
  = \uparrow u$;
  
  $X \sqsubseteq (\uparrow u)^{- 1} \circ \pi_A Y \circ \uparrow u
  \Leftrightarrow X \sqsubseteq (\uparrow \varphi)^{- 1} \circ
  \mathcal{E}_{\mathcal{C}}^{\Ob Y} \circ \pi_A Y \circ
  (\mathcal{E}_{\mathcal{C}}^{\Ob Y})^{- 1} \circ \uparrow \varphi
  \Leftrightarrow X \sqsubseteq (\uparrow \varphi)^{- 1} \circ
  \mathcal{E}_{\mathcal{C}}^{\Ob Y} \circ
  (\mathcal{E}_{\mathcal{C}}^{\Ob Y})^{- 1} \circ Y \circ
  \mathcal{E}_{\mathcal{C}}^{\Ob Y} \circ
  (\mathcal{E}_{\mathcal{C}}^{\Ob Y})^{- 1} \circ \uparrow \varphi
  \Leftrightarrow X \sqsubseteq (\uparrow \varphi)^{- 1} \circ Y \circ
  \uparrow \varphi \Leftrightarrow X \sqsubseteq (W f)^{- 1} \circ Y \circ W
  f$ what is true by definition of continuity.
\end{proof}

Equational definition of equalizers:

\url{http://nforum.mathforge.org/comments.php?DiscussionID=5328/}

\begin{thm}
  The following is an equalizer of parallel morphisms $f, g : A \rightarrow B$
  of category $\cont (\mathcal{C})$:
  \begin{itemize}
    \item the object $X = \iota_{\setcond{ x \in \Ob A }{
    f x = g x }} A$;
    
    \item the morphism $\mathcal{E}^{\Ob X, \Ob A}$ considered
    as a morphism $X \rightarrow A$.
  \end{itemize}
\end{thm}

\begin{proof}
  Denote $e = \mathcal{E}^{\Ob X, \Ob A}$.
  
  Let $f \circ z = g \circ z$ for some morphism $z$.
  
  Let's prove $e \circ u = z$ for some $u : \Src z \rightarrow X$.
  Really, as a morphism of $\mathbf{Set}$ it exists and is unique.
  
  Consider $z$ as as a generalized element.
  
  $f (z) = g (z)$. So $z \in X$ (that is $\Dst z \in X$). Thus $z = e
  \circ u$ for some $u$ (by properties of $\mathbf{Set}$). The
  generalized element $u$ is a $\cont (\mathcal{C})$-morphism
  because of the lemma above. It is unique by properties of
  $\mathbf{Set}$.
\end{proof}

We can (over)simplify the above theorem by the obvious below:

\begin{obvious}
$\setcond{ x \in \Ob A }{ f x = g x } = \dom (f \cap g)$.
\end{obvious}

\section{Co-equalizers}

\url{http://math.stackexchange.com/questions/539717/how-to-construct-co-equalizers-in-mathbftop}	

Let $\sim$ be an equivalence relation. Let's denote $\pi$ its canonical
projection.

\begin{defn}
  $f / \sim = \uparrow \pi \circ f \circ \uparrow \pi^{- 1}$ for every
  morphism $f$.
\end{defn}

\begin{obvious}
$\Ob (f / \sim) = (\Ob f) / r$.
\end{obvious}

\begin{obvious}
$f / \sim = \langle \uparrow^{\mathsf{FCD}} \pi \times^{(C)}
\uparrow^{\mathsf{FCD}} \pi \rangle f$ for every morphism
$f$.
\end{obvious}

To define co-equalizers of morphisms $f$ and $g$ let $\sim$ be is the smallest
equivalence relation such that $f x = g x$.

\begin{lem}
  Let $f : X \rightarrow Y$ be a morphism of the category
  $\cont (\mathcal{C})$ where $\mathcal{C}$ is a concrete
  category (so $W f = \uparrow \varphi$ for a $\mathbf{Rel}$-morphism
  $\varphi$ because $f$ is principal) such that $\varphi$ respects $\sim$.
  Factor it $\varphi = u \circ \pi$ where $u : \Ob (X / \sim)
  \rightarrow \Ob Y$ using properties of $\mathbf{Set}$. Then
  $u$ is a morphism of $\cont (\mathcal{C})$ (that is a
  continuous function $X / \sim \rightarrow Y$).
\end{lem}

\begin{proof}
  $f \circ X \circ f^{- 1} \sqsubseteq Y$; $\uparrow u \circ \uparrow \pi
  \circ X \circ \uparrow \pi^{- 1} \circ \uparrow u^{- 1} \sqsubseteq Y$;
  $\uparrow u \in \mathrm{C} (\uparrow \pi \circ X \circ \uparrow \pi^{- 1} ,
  Y) = \mathrm{C} (X / \sim , Y)$.
\end{proof}

\begin{thm}
  The following is a co-equalizer of parallel morphisms $f, g : A \rightarrow
  B$ of category $\cont (\mathcal{C})$:
  \begin{itemize}
    \item the object $Y = f / \sim$;
    
    \item the morphism $\pi$ considered as a morphism $B \rightarrow Y$.
  \end{itemize}
\end{thm}

\begin{proof}
  Let $z \circ f = z \circ g$ for some morphism $z$.
  
  Let's prove $u \circ \pi = z$ for some $u : Y \rightarrow \Dst z$.
  Really, as a morphism of $\mathbf{Set}$ it exists and is unique.
  
  $\Src z \in Y$. Thus $z = u \circ \pi$ for some $u$ (by properties of
  $\mathbf{Set}$). The function $u$ is a $\cont
  (\mathcal{C})$-morphism because of the lemma above. It is unique by
  properties of $\mathbf{Set}$ ($\pi$ obviously respects equivalence
  classes).
\end{proof}

\section{Rest}

\begin{thm}
  The categories $\cont (\mathcal{C})$ (for example in
  $\mathbf{Fcd}$ and $\mathbf{Rld}$) are complete.
  \fxwarning{Note that small complete category is a preorder!}
\end{thm}

\begin{proof}
  They have products and equalizers.
\end{proof}

\begin{thm}
  The categories $\cont (\mathcal{C})$ (for example in
  $\mathbf{Fcd}$ and $\mathbf{Rld}$) are co-complete.
\end{thm}

\begin{proof}
  They have co-products and co-equalizers.
\end{proof}

\begin{defn}
  I call morphisms $f$ and $g$ of a category with embeddings
  \emph{equivalent} ($f \sim g$) when there exist a morphism $p$ such that
  $\Src p \sqsubseteq \Src f$, $\Src p \sqsubseteq
  \Src g$, $\Dst p \sqsubseteq \Dst f$, $\Dst p
  \sqsubseteq \Dst g$ and $\iota_{\Src f, \Dst f} p = f$ and
  $\iota_{\Src g, \Dst g} p = g$.
\end{defn}

\begin{problem}
  Find under which conditions:
  \begin{enumerate}
    \item Equivalence of morphisms is an equivalence relation.
    
    \item Equivalence of morphisms is a congruence for our category.
  \end{enumerate}
\end{problem}
