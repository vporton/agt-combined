\documentclass[a4paper,oneside,english,reqno]{amsart}
% \usepackage[T1]{fontenc}
\usepackage{agt}

\hypersetup{pdftitle={On Ordered Semicategory Actions},
 pdfauthor={Victor Porton},
 pdfsubject={Ordered Semicategories Actions},
 pdfkeywords={ordered semicategories actions, algebra, universal algebra, general topology}}

\begin{document}

\title{On Ordered Semicategory Actions}

\author{Victor Porton, ORCID 0000-0001-7064-7975}

\email{\href{mailto:porton.victor@gmail.com}{porton.victor@gmail.com}}

\urladdr{\href{http://math.portonvictor.org}{http://math.portonvictor.org}}

\date{\today}

\begin{abstract}
I introduce \emph{ordered semicategory actions} and \emph{ordered semigroup actions} into mathematical vernacular with the purpose to use them in general topology. A wide class of ordered semigroup actions are embedded into an algebraic variety.
\end{abstract}

\keywords{ordered semicategories actions, algebra, universal algebra, general topology}

\subjclass[2020]{06A06, 06F05, 20M30, 20M99, 54B99}

\maketitle


Each definition containing the word ``semicategory'' will also implicity create a similar definition for semigroups.

\section{Category of ordered semicategory actions}

\begin{defn}
The category~$\mathbf{SemiCat}$ of \emph{semicategories} is small semicategories with semifunctors as morphisms.
\end{defn}

\begin{defn}
The category~$\mathbf{SemiCatAct}$ of \emph{semicategory actions} is small semicategories~$\mathcal{C}$ together with a semifunctor $\supfun{}:\mathcal{C}\to\mathbf{Set}$ with semifunctors as morphisms.
\end{defn}

\begin{defn}
\emph{Ordered semicategory} is a semicategory together with order on each of its $\Hom$-sets with the condition
\[ f_0\sqsubseteq f_1\land g_0\sqsubseteq g_1\Rightarrow g_0\circ f_0\sqsubseteq g_1\circ f_1. \]
\end{defn}

\begin{defn}
The category~$\mathbf{OrdSemiCat}$ of \emph{ordered semicategories} is small ordered semicategories with monotone semifunctors as morphisms.
\end{defn}

\begin{defn}
The category~$\mathbf{SemiOrdSemiCatAct}$ of \emph{semi-ordered semicategory actions} is:
\begin{description}
\item[\textbf{objects}] small ordered semicategories~$\mathcal{C}$ together with a semifunctor $\supfun{}:\mathcal{C}\to\mathbf{Pos}$ conforming to the identity $a\sqsubseteq b\Rightarrow\supfun{a}x\sqsubseteq\supfun{b}x$;
\item[\textbf{morphisms}] monotone semifunctors~$F$ between them together with a function~$G$ such that $\supfun{Fa}x=G\supfun{a}x$.
\end{description}

In other words, $\mathbf{OrdSemiCatAct}$ is small semi-ordered semicategories together with a function~$\supfun{}$ to $\mathbf{Pos}$ with morphisms being a monotone semifunctors~$F$ together with a function~$G$ such that conforming to the identities:
\begin{enumerate}
\item $\supfun{b\circ a}=\supfun{b}\circ\supfun{a}$;
\item $a\sqsubseteq b\Rightarrow\supfun{a}x\sqsubseteq \supfun{b}x$;
\item $a\sqsubseteq b\Rightarrow Ga\sqsubseteq Gb$;
\item $\supfun{Fa}x=G\supfun{a}x$.
\end{enumerate}

I call objects of this category \emph{semi-ordered semicategory actions}.
\end{defn}

\begin{defn}
\[ \mathbf{OrdSemiCatAct}\text{ is the wide subcategory of }\mathbf{SemiOrdSemiCatAct} \]
determined by the additional axiom $x\sqsubseteq y\Rightarrow\supfun{a}x\sqsubseteq\supfun{a}y$.

I call objects of this category \emph{ordered semicategory actions}.
\end{defn}

\begin{rem}
Contrary to sound mind, ordered semicategory actions (and ordered semigroup actions) were discovered only in 2019 (by me).
\end{rem}

\begin{defn}
I call objects of ordered semicategory actions \emph{interspaces} and elements of ordered semigroup actions \emph{spaces}.
\end{defn}

\begin{rem}
The usage of the word ``space'' in the above definition will be justified later.
\end{rem}

\section{Restricted identities}

Fix an ordered semicategory action.

\begin{defn}
\emph{Semicategory action with identities} is an ordered semicategory~$S$ action~$\supfun{}$ together with a function $p\mapsto\id_p\in\End(A)$ for $p\in A$ for $A\in\Ob S$ such that whenever the left or right side is defined: \[ \supfun{\id_p}x=x\sqcap p. \]

Morphisms $\id_p$ are called \emph{restricted identities}.
\end{defn}

\begin{rem}
The following axioms often hold, too, but are not needed below:
\begin{enumerate}
\item $\id_p\circ f\sqsubseteq f$;
\item $f\circ\id_p\sqsubseteq f$;
\item $\id_p\circ\id_q=\id_{p\sqcap q}$;
\item $p\sqsubseteq q\Rightarrow\id_p\sqsubseteq\id_q$;
\item $\id_p\sqsubseteq 1$;
\item $\id_{\top}=1$.
\end{enumerate}
\end{rem}

\begin{prop}
\label{prop:id-inj}
$x\mapsto\id_x$ is an injection.
\end{prop}

\begin{proof}
Let $x\ne y$. Then $x\ne x\sqcap y$ (possibly, $x\sqcap y$ is undefined).
\[ \supfun{\id_y}x=x\sqcap y\ne x=x\sqcap x=\supfun{\id_x}x. \]
\end{proof}

\begin{defn}
Morphisms of the \emph{category of semicategory actions with identities.} are morphisms $(F,G)$ of the category of ordered semicategory actions conforming to the identity $F\id_p=\id_{Gp}$.
\end{defn}

\begin{prop}
For a $\sqcap$-se\-mi\-lat\-ti\-ce semicategory action, the semilattice axioms can be rewritten as:
\begin{enumerate}
\item $\supfun{\id_a}\supfun{\id_b}c = \supfun{\id_{\supfun{\id_a b}}}c$;
\item $\supfun{\id_a}b = \supfun{\id_b}a$;
\item $\supfun{\id_a}a = a$.
\end{enumerate}
\end{prop}

\emph{Restriction} of an interspace~$a$ to element~$x$ is $a|_x=a\circ\id_x$.

\emph{Square restriction} (a generalization of restriction of a topological space, metric space, etc.) of a space~$a$ to element~$x$ is $\id_x\circ a\circ\id_x$.

\section{Actionable semigroups}

\begin{defn}
An \emph{actionable semicategory} is a semicategory, and $\cdot$ is a binary operation on $A$ such that $(g\circ f)\cdot x = g\cdot(f\cdot x)$.
\end{defn}

\begin{defn}
An \emph{ordered actionable semicategory} is an actionable semicategory which is also an ordered semicategory with additional equation:
\begin{enumerate}
\item $f_0\sqsubseteq f_0\land g_1\sqsubseteq g_0 \Rightarrow g_0\cdot f_0\sqsubseteq g_1\cdot f_1$.
\end{enumerate}
\end{defn}

\begin{defn}
An \emph{actionable semicategory with identities} is an actionable semicategory that is also a semicategory with identities.
\end{defn}

\begin{defn}
The category $\mathbf{Act^{\ast}SemiCat}$ of actionable semicategories is defined as usual in universal algebra.
\end{defn}

\begin{defn}
The morphisms of the category $\mathbf{Act^{\ast}OrdSemiCat}$ of ordered actionable semicategories are semifunctors that are both morphisms of $\mathbf{Act^{\ast}SemiCat}$ and $\mathbf{Pos}$ under the conditions that:
\[
f_0\sqsubseteq f_1\land g_0\sqsubseteq g_1\Rightarrow f_0\cdot g_0\sqsubseteq f_1\cdot g_1.
\]
\end{defn}

\begin{defn}
The morphisms of the category $\mathbf{Act^{\ast}SemiCatIden}$ of actionable semicategories with identities are semifunctors that are both morphisms of $\mathbf{Act^{\ast}OrdSemiCat}$ under the conditions that: ??
\end{defn}

\begin{obvious}
An actionable semigroup is an algebraic variety.
\end{obvious}

\begin{obvious}
Actionable ordered semigroups and actionable semigroups with identities are relational algebras.
\end{obvious}

\section{From actionable semicategory to semicategory action}

\begin{defn}
If $(\dots,\circ,\cdot)$ is an actionable semicategory, then we can define the operation $\supfun{}$ by $\supfun{f}x = f\cdot x$.
\end{defn}

\begin{obvious}
The above defines an semicategory action with identities with operations~$\circ$ and~$\supfun{}$.
\end{obvious}

\begin{rem}
Semicategory actions defined this way have the sets of objects and morphisms equal to each other.
\end{rem}

\begin{obvious}
If it is an ordered actionable semicategory [??defined], then the result is a semicategory action with identities.
\end{obvious}

\begin{obvious}
If it is an actionable semicategory with identities [??defined], then the result is an ordered semicategory action with identities.
\end{obvious}

\section{From semicategory action with identities to actionable semicategory}

Let $(A,\circ,\supfun{},p\mapsto\id_p)$ be an ordered semicategory action with identities, each object set having greatest element~$\top$.

\begin{obvious}
$\supfun{f}x = \supfun{f}\supfun{\id_x}\top = \supfun{f\circ\id_x}\top = \im(f\circ\id_x)$.
\end{obvious}

By analogy with the above obvious, we can define $g\cdot f = \id_{\im(g\circ f)}$.

\begin{prop}
The above with the same semigroup operation~$\circ$ defines an actionable semicategory.
\end{prop}

\begin{proof}
Really, $(g\circ f)\cdot x = \id_{\im(g\circ f\circ x)} = \id_{\supfun{g}\supfun{f}\supfun{x}\top} = \id_{\supfun{g}{\im(f\circ x)}} = \id_{\supfun{g}(\top\sqcap\im(f\circ x))} = \id_{\im(g\circ(y\mapsto y\sqcap\im(f\circ x))} = \id_{\im(g\circ\id_{\im(f\circ x)})} = g\cdot(f\cdot x)$.

Monotonicity is obvious.
\end{proof}

\begin{rem}
Identities can't be the same, because the set of objects is not the same.
\end{rem}

\section{An injection}

Fix a semicategory action with identities $(\mathcal{C},\supfun{},p\mapsto\id_p)$, where $\mathcal{C}$ is an ordered semicategory.

Then \[ g\cdot\id_x = \supfun{g}\id_x = \id_{\im(g\circ\id_x)} = \id_{\supfun{g}x}. \]

So, we extended the mapping $\supfun{g}$ to mapping $g\cdot'f=\supfun{g}'f=\id_{\im(g\circ f)}$ on a wider set, because $x\mapsto\id_x$ is an injection (see proposition~\ref{prop:id-inj}).

\begin{obvious}
$(\mathcal{C},\cdot')$ is an ordered actionable semicategory, corresponding (as above) to the ordered semicategory action $(\mathcal{C},\supfun{}')$.
\end{obvious}

\begin{prop}
$\supfun{\id_p}'f = \id_{p\sqcap\im f}$.
\end{prop}

\begin{proof}
$\supfun{\id_p}'f = \id_{\im(\id_p\circ f)} = \id_{\supfun{\id_p}\supfun{f}\top} = \id_{p\sqcap\im f}$
\end{proof}

\begin{thm}
$(\mathcal{C},\supfun{}'|_{\setcond{\id_x}{x\in\mathcal{C},\mathcal{C}\in\Ob\mathcal{C}}})$ is an ordered semicategory action isomorphic (by an isomorphism $(1,p\mapsto\id_p)$ in $\mathbf{OrdSemiCatAct}$) to the original one $(\mathcal{C},\supfun{})$.
\end{thm}

\begin{proof}
Denote $\Phi$ the functor that maps an object $(\mathcal{C},\supfun{})$ to $(\mathcal{C},\supfun{}')$ and a morphism $(1,G)$ to $(1,p\mapsto\id_p)$ and $\Psi$ the unique functor that maps an object $(\mathcal{C},\supfun{}')$ to $(\mathcal{C},\supfun{})$ and a morphism $(1,G)$ to $(1,\id_p\mapsto p)$ ($p\mapsto\id_p$ is an injection). It's obvious, that $\Phi$ and $\Psi$ are inverse to each other.
\end{proof}

We can also restrict it to restricted identities: $\supfun{f}\id_x = f\cdot\id_x$ and obtain this way an ordered semicategory action isomorphic to the original one.

\section{idempotency of our ``inducing'' operations}

Every semicategory action with identities, whose objects have greatest elements, induces an actionable semicategory with identities, which induces another semicategory action with identities. Thus every semigroup with identities, whose objects have greatest elements, induces another semigroup with identities. The formula for this is $\supfun{g}'f = \id_{\im(g\circ f)}$, where $\supfun{}'$ is the induced action.

\begin{prop}
The mapping on semicategory action with identities, defined by the previous paragraph, is idempotent.
\end{prop}

\begin{proof}
Let apply this formula two times:
\[ \supfun{g}''f = \id_{\im'(g\circ f)} = \id_{\supfun{g\circ f}'\top} = \id_{\im(g\circ f\circ\top)} = \id_{\im(g\circ f)} = \supfun{g}'f. \]
\end{proof}

\section{Actionable semigroup actions as algebraic varieties}

\begin{rem}
It is even a universal algebra variety\footnote{See here for \href{https://chatgpt.com/s/t_691e8f88da64819184591f9e310e1728}{properties of categories isomorphic to universal algebras}.} in the case of $\sqcap$-semilattice.
\end{rem}

\begin{obvious}
The universal algebra signature for actionable semigroup with identities that is a $\sqcap$-semilattice:

Operations $({\circ}, {\cdot}, p\mapsto\id_p)$ with axioms:
\begin{enumerate}
\item $h\circ(g\circ f) = (h\circ g)\circ f$ (semigroup);
\item $h\cdot(g\cdot f) = (h\circ g)\cdot f$ (actionable);
\item $\id_a\cdot(\id_b\cdot c) = \id_{\id_a\cdot b}\cdot c$ ($\sqcap$-semilattice)\footnote{$a\sqcap(b\sqcap c) = (a\sqcap b)\sqcap c$};
\item $\id_a\cdot b = \id_b\cdot a$ (ditto)\footnote{$a\sqcap b = b\sqcap a$};
\item $\id_a\cdot a = a$ (ditto)\footnote{$a\sqcap a=a$};
\item $\id_{g_0\circ(\id_{\id_{f_0}\cdot f_1})}\cdot((\id_{g_0}\cdot g_1)\circ(\id_{f_0}\cdot f_1)) = (\id_{g_0}\cdot g_1)\circ(\id_{f_0}\cdot f_1)$ (ordered semigroup)\footnote{$(g_0\circ(f_0\sqcap f_1))\sqcap((g_0\sqcap g_1)\circ(f_0\sqcap f_1)) = (g_0\sqcap g_1)\circ(f_0\sqcap f_1)$};
\item $\id_{((\id_{f_1}\cdot f_0)\cdot g_0)}\circ((\id_{f_0}\cdot f_1)\cdot(\id_{g_0}\cdot g_1)) = (\id_{f_0}\cdot f_1)\circ(\id_{g_0}\cdot g_1)$ (ditto)\footnote{$((f_0\sqcap f_1)\circ g_0)\sqcap((f_0\sqcap f_1)\circ(g_0\sqcap g_1)) = (f_0\sqcap f_1)\circ(g_0\sqcap g_1)$};
\item $\id_{g_0\cdot(\id_{\id_{f_0}\cdot f_1})}\cdot((\id_{g_0}\cdot g_1)\cdot(\id_{f_0}\cdot f_1)) = (\id_{g_0}\cdot g_1)\cdot(\id_{f_0}\cdot f_1)$ (ordered actionable semigroup)\footnote{$(g_0\cdot(f_0\sqcap f_1))\sqcap((g_0\sqcap g_1)\cdot(f_0\sqcap f_1)) = (g_0\sqcap g_1)\cdot(f_0\sqcap f_1)$};
\item $\id_{((\id_{f_1}\cdot f_0)\cdot g_0)}\cdot((\id_{f_0}\cdot f_1)\cdot(\id_{g_0}\cdot g_1)) = (\id_{f_0}\cdot f_1)\cdot(\id_{g_0}\cdot g_1)$ (ditto)\footnote{$((f_0\sqcap f_1)\cdot g_0)\sqcap((f_0\sqcap f_1)\cdot(g_0\sqcap g_1)) = (f_0\sqcap f_1)\cdot(g_0\sqcap g_1)$}.
\end{enumerate}
\end{obvious}

\section{Significance of ordered semicategory actions}

My research (mainly in~\cite{volume-1}) shows that all kinds of spaces met in general topology:
\begin{itemize}
\item (pre-)topological spaces;
\item (quasi-)metric spaces;
\item (quasi-)proximity spaces;
\item (quasi-)uniform spaces;
\item locales and frames;
\item Cauchy spaces
\end{itemize}
are embedded into ordered semicategory actions with identities and therefore are embedded into the algebra variety of actionable ordered semigroups. That is all general topology is an algebra, moreover all of it is the same algebraic variety (yes, even metric spaces).

However, because \cite{volume-1}~is currently a draft, I am still ``confused'' about what kind of category embeddings they are: category isomorphism, category equivalence, or like this.

I will publish the details later.

In any case, that the entire theory of general topology is reducible to ordered semicategory actions, makes ordered semicategory actions immensely important, about on-par with group theory.

It is yet unclear for me, whether non-ge\-ne\-ral-to\-po\-lo\-gy spaces such as Euclidean spaces are also reducible to ordered semigroup actions, so possibly defining ``space in general''.

\bibliographystyle{plain}
\bibliography{refs}

\end{document}