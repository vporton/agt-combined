\documentclass{amsart}
\usepackage{stmaryrd}

\usepackage{agt}

\begin{document}

This document contains a list of short ideas of future research in Algebraic
General Topology.

I have created branch \texttt{devel} in \href{https://bitbucket.org/portonv/algebraic-general-topology}{the \LaTeX repository} for the book
to add new ``draft'' features there. The \texttt{devel} branch isn't distributed by me in PDF format, but you can download and compile it yourself.

This research plan is not formal and may contain vague statements.

\section{Things to do first}

Isn't generalized limit just the limit on the set of ``singularities''? If yes, it seems a key to put it into a diffeq!

Which filter operations are congruences on
equivalence of filters?

\section{Misc}

Consider a semigroup action defined by $\supfun{a}x=a\circ x$.
Represent ordered semi- group/category actions algebraically by representing
a semigroup element~$x$ as a constant function~$d\times x$.
$f\circ\varphi x=\varphi y$, if $\varphi$ is an isomorphism defines
an action of semigroup. Not clear which applications this have.
Apparently, not every action of ordered semigroup is defined this way.
Thus we have three semigroups: the orignial one and two defined by the formulas
$\supfun{a}x=a\circ x$ and $f\circ\varphi x=\varphi y$. It is expected that
for many properties these are equivalent.

$x\suprel{f}y$ is the same as $(x,f,y)$ being an element of a certain filter~$\varphi$
on triples. We can extend to quadruples and further by the formula
$(a,b,c,d)\in\varphi\Leftrightarrow a\suprel{c\circ b}d$. By the way,
composition is defined by these two filters.
$a\suprel{b}c\Leftrightarrow c\suprel{z}b$ for $z=(x\mapsto a\times x,x\mapsto\supfun{x}a$. So, we have a permutation of $(a,b,c)$. Try to use it to
solve complete distributivity of composition with a complete funcoid.


Define for OPA $\suprel{}$ by the formula $\suprel{g\circ a\circ f}=\supfun{g}\circ\suprel{a}\circ\supfun{f^{-1}}$.

Represent covers of~$A$ as funcoids on~$\subsets A$.

Funcoids can also be defined by the formula $Y\nasymp\alpha X\Leftrightarrow X\nasymp\beta Y$ (for sets) - is it known?
Equivalent: $Y\in\partial\alpha X\Leftrightarrow X\in\partial\beta Y$; $X\mathrel{\partial\alpha}Y\Leftrightarrow Y\mathrel{\partial\beta}X$.
In other words, funcoids can be defined as functions from sets to free stars, whose relational inverse is also a function from sets to free stars.
In other words, it is an arbitrary binary relation between sets each whose (both $x$ and $y$) projections are free stars.
(This idea simplifies the definion of prestaroids.)
In yet other words, it is an arbitrary binary relation between sets each whose (both $x$ and $y$) projections are ideals.
In yet other words, it is an arbitrary binary relation between sets each whose (both $x$ and $y$) projections are filters.
It can be generalized to point-free.

Both uniform covers and functions can be represented as sets of binary Cartesian products
(uniform covers as sets of ``quadrtatic'' products, function as sets of products of singletons).
Define composition as ??.
Therefore we can form a semigroup of them. What is the action of this semigroup?

Some special cases of reloids:
\url{https://www.researchgate.net/publication/331776637_Functional_Boundedness_of_Balleans_Coarse_Versions_of_Compactness}

``Unfixed'' for more general settings than lattice and its
sublattice. (However, it looks like this generalization has
no practical applications.)

Should clearly denote $\mathsf{pFCD}(\mathfrak{A};\mathfrak{Z})$ or $\mathsf{pFCD}(\mathfrak{A})$.

\url{https://en.wikipedia.org/wiki/Compact_element}

\url{https://arxiv.org/abs/1904.12525} On proximal fineness of topological groups in their right uniformity

\url{https://arxiv.org/abs/1905.00513}
On $\mathcal{B}$-Open Sets

\url{https://arxiv.org/abs/1812.09802} Boundaries of coarse proximity spaces and boundaries of compactifications

Try to describe a filter with up of infintiely small
components.
For this use a filter (of sets or filters) rather than a set
of sets.

About generalization of simplical sets for nearness spaces on posets?
\url{https://arxiv.org/abs/1902.07948}

\section{Category theory}

Can product morphism (in a category with restricted
identities) be considered as a categorical product
in \href{https://en.wikipedia.org/wiki/Comma_category#Arrow_category}{arrow category}?
(It seems impossible to define projections for
arbitrary categories with binary product morphism.
Can it be in the special cases of funcoids and reloids?)

Attempting to extend Tychonoff product from topologies to funcoids:
|| If $i$ has left adjoint:
If $r$ is left adjoint to $i$, we have $\Hom(A, i(X\times Y)) = \Hom(r(A), X\times Y) = \Hom(r(A), X)\times \Hom(r(A), Y) = \Hom(A, i(X))\times \Hom(A, i(Y))$.
|| If also the left adjoint is full and faithful:
$\Hom(A, i(r(X)\times r(Y))) = \Hom(r(A), r(X)\times r(Y)) = \Hom(r(A), r(X))\times \Hom(r(A), r(Y)) = \Hom(A, X)\times \Hom(A, Y)$.
See also \url{http://math.stackexchange.com/q/1982931/4876}. However this does not apply because reflection of topologies in funcoids
is not full.

\emph{Being intersecting} is defined for posets (=~thin categories). It seems that this can be generalized for any categories.
This way we can define (pointfree) funcoids between categories generalizing pointfree funcoids between posets.
(However this is probably easily reducible to the case of posets.)

I have defined $\mathsf{RLD}\sharp$ to describe $\Hom$-sets of the category or reloids but without source and destination and without composition.
$\mathsf{RLD}$ should be replaced with $\mathsf{RLD}\sharp$ where possible, in order to make the theorems throughout the book a little more general.
Also introduce similar features like $\Gamma\sharp$ and $\mathfrak{F}\Gamma\sharp$ (the last notation may need to be changed).

Misc properties of continuous functions between endofuncoids and endoreloids.

\url{http://nforum.ncatlab.org/discussion/6765/please-help-with-a-proof-that-a-category-is-monoidal/} proves that
finitary staroids are isomorphic to an ideal on a poset (for semilattices only).

\cite{filt-cat}~defines two categories with objects being filters. Another article on the same topic:\\
\url{https://eudml.org/doc/16352} (Koubek, V\'aclav, and Reiterman, Jan. "On the category of filters.")

\fxnote{\url{https://en.wikipedia.org/wiki/Cauchy space} says ``The category of Cauchy spaces and Cauchy continuous maps is cartesian closed.'' Generalize.
\url{http://www.sciencedirect.com/science/article/pii/0166864187900988}}

\section{Compact funcoids}

Generalize the theorem that compact topology corresponds to only one uniformity.

For compact funcoids the Cantor's theorem that a function continuous on a compact is uniformly continuous.

Every closed subset of a compact space is compact. A compact subset of a Hausdorff space is closed. 17.5 theorem in Willard.

17.6 theorem in Willard.

17.7 theorem in Willard: The continuous image of a compact space is compact.

17.10 Theorem in Willard: A compact Hausdorff space $X$ is a $T 4$-space. Also 17.11 Corollary, 17.13, 17.14 theorem.

"Locally compact" for funcoids. See also 18 "Locally compact spaces: in Willard.

Compactification.

\section{Misc}

It is easy to prove that
$f\circ(\mathcal{X}\times^{\mathsf{FCD}}\mathcal{Y})=\mathcal{X}\times^{\mathsf{FCD}}\supfun{\tofcd f}\mathcal{Y}$.
Use it to prove $f\circ\bigsqcup K=\bigsqcup_{g\in K}(f\circ g)$ for complete reloid~$f$.

A reloid~$f$ is complete iff $f\circ\bigsqcup K=\bigsqcup{g\in K}(f\circ g)$ for every set~$K$ of reloids.

A funcoid or pointfree funcoids can be turned into an ordered semigroup action also by the formula: $\supfun{f}(x,y)=(\supfun{f}(x),\supfun{f^{-1}}(y))$ (on the left $\supfun{}$ denotes the semigroup action, on the right it denotes components of the funcoid.) Can we similarly consider multifuncoids as an ordered semigroup action?

Counterexample at \url{https://math.stackexchange.com/a/3046071/4876}.

\url{https://www.researchgate.net/project/Contra-continuity-in-its-different-aspects}

We know that $\torldout(f\sqcup g) = \torldout f\sqcup\torldout g$.
Hm, then it is a pointfree funcoid!

\begin{conjecture}
$\langle f \rangle \bigsqcup S = \bigsqcup_{\mathcal{X} \in S} \langle f
\rangle \mathcal{X}$ if $S$ is a totally ordered (generalize for a filter
base) set of filters (or at least set of sets).
[Counterexample: \url{https://portonmath.wordpress.com/2018/05/20/a-counterexample-to-my-recent-conjecture/}]
\end{conjecture}

Should we replace the word ``intersect'' with the word ``overlap''?

Instead of a filtrator use ``closure`` $(X,[X])$?

$\tofcd$, $\torldin$, $\torldout$ can be defined purely in terms of filtrators.
So generalize it.

Generalize for funcoids and reloids factoring into monovalued and injective:\\
\url{https://math.stackexchange.com/q/2414159/4876}.
Generalize it for star-composition with multidimensional, identity relations, identity staroids/multifuncoids, or identity reloid.
Isn't thus a category with star-morphisms determined by a regular category?!
Also try to split into complete and co-complete funcoids/reloids.

Open problems on $\beta\omega$ (Klass Pieter Hart and Jab van Mill).

Example that $\Compl f\sqcup\CoCompl f\sqsubset f$ (for both funcoids and reloids).
Proof for funcoids (for reloids it's similar): Take $f=\mathcal{A}\times^{\mathsf{FCD}}\mathcal{B}$. Then (write an explicit proof)
$\Compl f=(\Cor\mathcal{A})\times^{\mathsf{FCD}}\mathcal{B}$ and $\CoCompl f=\mathcal{A}\times^{\mathsf{FCD}}(\Cor\mathcal{B})$.
Thus $\Compl f\sqcup\CoCompl f\ne f$ (if $\mathcal{A}$, $\mathcal{B}$ are non-principal).

Every funcoid (reloid) is a join of monovalued funcoids (reloids). For funcoids it's obvious
(because it's a join of atomic funcoids). For reloids?

``Vicinity'' and ``neighborhood'' mean different things, e.g. in \cite{converg}.

Micronization~$\mu$ and $S^\ast$ are in some sense related as Galois connection.
To formalize this we need to extend $\mu$ to arbitrary reloids (not only binary relations).

We need (it is especially important for studying compactness) to find a product of funcoids which coincides with
product of topological spaces. (Cross-composition product doesn't because it is even not a funcoid (but a pointfree funcoid).)
Neither subatomic product.

Subspace topology for space~$\mu$ and set~$X$ is equal to $\mu\sqcap(X\times^{\mathsf{FCD}} X)$.

Change terminology: \emph{monotone} $\rightarrow$ \emph{increasing}.

What are necessary and sufficient conditions for $\up f$ to be a filter for a funcoid~$f$?

Article ``Neighborhood Spaces'' by D. C. KENT and WON KEUN MIN.
\url{ftp://ftp.math.ethz.ch/EMIS/journals/IJMMS/Volume32 7/239107.pdf}

$g\sqsubseteq f^{\circ}\circ h \Leftrightarrow f\circ g\sqsubseteq h$?

$\lim {x\rightarrow a} f(x) = b$ iff $x\rightarrow a$ implies $\supfun{f}x\rightarrow b$ for all filters~$x$.

\url{http://mathoverflow.net/q/36999/4086} ``A good place to read about uniform spaces''.

Research the posets of all proximity spaces and all uniform spaces (and also possibly reflexive and transitive funcoids/reloids).

Are filters on all Heyting or all co-Heyting lattices star-separable?
\url{http://math.stackexchange.com/q/1326266/4876}

Define generalized pointfree reloids as filters on systems of sides.

\href{https://www.math.ksu.edu/~strecker/primer.ps}{Galois connections primer} -- study to ensure that we considered all Galois connections properties.

\href{https://en.wikipedia.org/wiki/Germ (mathematics)}{Germs} seems to be equivalent to monovalued reloids.

$\mathcal{A} = \min\setcond{X}{\forall K\in\corestar\mathcal{A}:K\nasymp K}$, so we can restore $\mathcal{A}$ from~$\corestar\mathcal{A}$.

Boolean funcoid is a join-semilattice morphism from a boolean lattice to a boolean lattice. Generalize for pointfree funcoids.

Another way to define pointfree reloid as filters on Galois connections between two posets.

$L \in \GR \prod^{\mathsf{Strd} \ast} A \Leftrightarrow \forall
\text{finite } M \subseteq \dom A \forall i \in M : A i \nasymp L i$?

Star-composition with identity staroids?

Does upgrading/downgrading of the ideal which represents a prestaroid coincide with upgrading/downgrading of the prestaroid?

It seems that equivalence of filters on different bases can be generalized:
filters~$\mathcal{A}\in\mathfrak{A}$ and~$\mathcal{B}\in\mathfrak{B}$ are \emph{equivalent} iff
there exists an $X\in\mathfrak{A}\cap\mathfrak{B}$ which is greater than both~$\mathcal{A}$ and~$\mathcal{B}$.
This however works only in the case if order of the orders~$\mathfrak{A}$ and~$\mathfrak{B}$ agree,
that is if then are both a suborders of a greater fixed order.

Under which conditions a function spaces of posets is strongly separable?

Generalize both funcoids and reloids as filters on a superset of the lattice~$\Gamma$ (see ``Funcoids are filters'' chapter).

When the set of filters closed regarding a funcoid is a (co-)frame?

If a formula $F(x 0,\dots,x n)$ holds for every poset $\mathfrak{A} i$ then it also holds for product order $\prod\mathfrak{A}$.
(What about infinite formulas like complete lattice joins and meets?)
Moreover $F(x 0,\dots,x n) = \mylambda{i}{n}{F(x {0,i},\dots,x {n,i})}$ (confused logical forms and functions).
It looks like a promising approach, but how to define it exactly? For example, $F$ may be a form always true for boolean
lattices or for Heyting lattices, or whatsoever. How one theorem can encompass all kinds of lattices and posets?
We may attempt to restrict to (partial) functions determined by order.
(This is not enough, because we can define an operation restricting $\setminus$ defined only for posets
of cardinality above or below some cardinal~$\kappa$. For such restricted $\setminus$ the above formula does not work.)
See also \url{https://portonmath.wordpress.com/2016/01/12/a-conjecture-about-product-order-and-logic/}.
It seems that \noun{Todd Trimble} shows a general category-theoretic way to describe this:
\url{https://nforum.ncatlab.org/discussion/6887/operations-on-product-order/}.

Get results from \url{http://ncatlab.org/toddtrimble/published/topogeny}.

What about distributivity of quasicomplements over meets and joins for the filtrator of funcoids? Seems like nontrivial conjectures.

Conjecture: Each filtered filtrator is isomorphic to a primary filtrator. (If it holds, then primary and filtered filtrators are the same!)

Add analog of the last item of the theorem about co-complete funcoids for pointfree funcoids.

Generalize theorems about $\mathsf{RLD}(A;B)$ as $\mathscr{F}(A\times B)$ in order to clean up the notation
(for example in the chapter ``Funcoids are filters'').

Define reloids as a filtrator whose core is an ordered semigroup.
This way reloids can be described in several isomorphic ways (just like primary filtrators are both filtrators of filters, of ideals, etc.)
Is it enough to describe all properties of reloids? Well, it is not a semigroup, it is a precategory.
It seems that we also need functions $\dom$ and $\im$ into partially ordered sets and ``reversion'' (dagger).

\url{http://mathoverflow.net/a/191381/4086} says that $n$-staroids can be identified with certain ideals!

To relax theorem conditions and definition, we can define \emph{protofuncoids} as arbitrary pairs $(\alpha;\beta)$ of functions
between two posets. For protofuncoids composition and reverse are defined.

Add examples of funcoids to demonstrate their power:
$D\sqcup T$ ($D$ is a digraph $T$ is a topological space),
$T\sqcap\setcond{(x;y)}{y\ge x}$ as ``one-side topology'' and also a circle made from its $\pi$-length segment.

Say explicitly that pseudodifference is a special case of difference.

For pointfree funcoids, if $f:\mathfrak{A}\rightarrow\mathfrak{B}$ exists,
then existence of least element of~$\mathfrak{A}$ is equivalent to existence of least element of~$\mathfrak{B}$:
$y \nasymp \supfun{f} \bot^{\mathfrak{A}} \Leftrightarrow
\bot^{\mathfrak{A}} \nasymp \langle f^{- 1} \rangle y \Leftrightarrow 0$. Thus
$\supfun{f} y \asymp \supfun{f} y$ and so $\supfun{f} y =
\bot^{\mathfrak{B}}$.
Can a similar statement be made that $\mathfrak{A}$ being join-semilattice implies $\mathfrak{B}$ being join-semilattice
(at least for separable posets)? If yes, this could allow to shorten some theorem conditions.
It seems we can produce a counter-example for non-separable posets by replacing an element with another element with the same full star.

Develop Todd Trimble's idea to represent funcoids as a relation~$\xi$ further:
Define funcoid as a function from sets to sets of sets
$\xi(A\sqcup B) = \xi A\cap\xi B$ and $\xi\bot=\emptyset$.

Denote the set of least elements as $\operatorname{Least}$. (It is either an
one-element set or empty set.)

Show that cross-composition product is a special case of infimum product.

Analog of order topology for funcoids/reloids.

A set is connected if every function from it to a discrete space is constant. Can this be generalized for generalized connectedness and generalized continuity? I have no idea how to relate these two concepts in general.

Develop theory of \emph{funcoidal groups} by analogy with topological groups.
Attempt to use this theory to solve this open problem:\\
\url{http://garden.irmacs.sfu.ca/?q=op/is every regular paratopological group tychonoff}
Is it useful as topological group determines not only a topology but even a uniformity?
An interesting article on topological groups:
\url{https://arxiv.org/abs/1901.01420}

Consider generalizations of this article: \\
\url{https://www.researchgate.net/publication/318822240_Categorically_Closed_Topological_Groups}

A space $\mu$ is $T 2$- iff the diagonal $\Delta$ is closed in $\mu\times\mu$.

The $\beta$-th projection map is not only continuous but also open (Willard, theorem 8.6).

$T x$-separation axioms for products of spaces.

Willard 13.13 and its important corollary 13.14.

Willard 15.10.

About real-valued functions on endofuncoids: Urysohn's Lemma (and consequences: Tietze's extension theorem) for funcoids.

About product of reloids:\\
\url{http://portonmath.wordpress.com/2012/05/23/unfirunded-questions/}

Generalized Fr\'echet filter on a poset (generalize for filtrators) $\mathfrak{A}$ is a filter $\Omega$ such that
\[ \corestar\Omega = \setcond{x\in\mathfrak{A}}{\atoms x\text{ is infinite}}. \]
Research their properties (first, whether they exists for every poset).\
Also consider Fr\'echet element of $\mathsf{FCD}(A;B)$.
Another generalization of Fr\'echet filter is meet of all coatoms.

Manifolds.

\url{http://www.sciencedirect.com/science/article/pii/0304397585900623}\\
(free download, also Google for "pre-adjunction", also "semi" instead of "pre") Are $\tofcd$ and $\torldin$ adjunct?

Check how \href{http://ncatlab.org/nlab/show/multicategory}{multicategories}
are related with categories with star-morphisms.

At \url{https://en.wikipedia.org/wiki/Semilattice} they are defined distributive
semilattices. A join-semilattice is distributive if and only if the lattice of its ideals (under inclusion) is distributive.

The article \url{http://arxiv.org/abs/1410.1504} has solved ``Every paratopological group is Tychonoff'' conjecture positively.
Rewrite this article in terms of funcoids and reloids (especially with the algebraic formulas characterizing regular funcoids).

Generalize interior in topological spaces as the \emph{interior funcoid} of a co-complete funcoid~$f$, defined as a pointfree funcoid
$f^\circ: \mathscr{F}\dual\Src f \rightarrow \mathscr{F}\dual\Dst f$ conforming to the formula:
$\rsupfun{f^{\circ}} (I \sqcap J) = \overline{\rsupfun{f} \overline{I \sqcap J}} = \overline{\rsupfun{f} ( \overline{I} \sqcup \overline{J} )}$.
However composition of an interior funcoid with a funcoid is neither a funcoid nor an interior funcoid.
It can be generalized using pseudocomplement.

\url{http://math.sun.ac.za/cattop/Output/Kunzi/quasiintr.pdf} ``An Introduction to the Theory of Quasi-uniform Spaces''.

\url{http://www.mscand.dk/article/download/10581/8602} (``On equivalence between proximity structures and totally bounded uniform structures'')

Characterize the set $\setcond{f\in\mathsf{FCD}}{\torldin f=\torldout f}$. (This seems a difficult problem.)
Another (possibly related) problem: when $\up f$ is a filter for a funcoid~$f$.

Define $S^\ast (f)$ for a funcoid~$f$ (using that~$f$ is a filter).

Let $\mathcal{A}$ be a filter. Is the boolean algebra~$Z(D\mathcal{A})$
a. atomic; b. complete?

\url{https://arxiv.org/pdf/1003.5377.pdf}

\url{https://www.researchgate.net/project/Generalized-topological-groups-in-Delfs-Knebusch-generalized-topology}
and
\url{http://www.sciencedirect.com/science/authShare/S0166864117302742/20170530T163200Z/1?md5=a5f9bcce5a6c49d4b8b35fdc0d2f9105}
(not available for free).

\url{https://arxiv.org/abs/1802.05746} about uniform spaces and
function spaces.

\url{https://arxiv.org/abs/1904.08969} about $k$-Scattered spaces.

\url{https://www.researchgate.net/publication/333731858_SUPERCOMPACT_MINUS_COMPACT_IS_SUPER}
seems interesting.

``Second reloidal product'' of more than two filters.
Also starred second product.

Homotopy with a monovalued (complete?) funcoid from~$\mathbb{R}$
instead of path.

What's about limits multidimesional functions? $\forall x_i : x_i \rightarrow
\alpha_i \Rightarrow f (x) \rightarrow \beta$.

``Contra continuity'' (see journals).

Product of pointfree funcoids considered as structures:
\url{http://citeseerx.ist.psu.edu/viewdoc/download?doi=10.1.1.414.9364&rep=rep1&type=pdf}.
$2$-staroids are universal classes: \url{https://modeltheory.fandom.com/wiki/Universal_theory}

\url{http://imar.ro/journals/Revue_Mathematique/pdfs/2015/2/5.pdf}
``ON IDEALS AND FILTERS IN POSETS'' SERGIU RUDEANU.

\url{https://www.researchgate.net/publication/334695188_A_note_on_compact-like_semitopological_groups}
A note on compact-like semitopological groups.

\url{https://arxiv.org/abs/1907.12129} Closed subsets of compact-like topological spaces.

\url{https://arxiv.org/abs/1908.05624} A remark on locally direct product subsets in a topological Cartesian space.

\url{https://arxiv.org/abs/1909.06428} Coproducts of proximity spaces.

\url{https://arxiv.org/abs/1909.09303} On $T_0$ spaces determined by well-filtered spaces.

\url{https://arxiv.org/abs/1812.06379} Closed Discrete Selection in the Compact Open Topology.

\url{https://www.researchgate.net/publication/333731858_SUPERCOMPACT_MINUS_COMPACT_IS_SUPER}
Supercompact minus compact is super.

\url{https://www.jstor.org/stable/2306387?read-now=1&seq=1} Ideals in partially ordered sets (free to read).

\url{https://blog.mathematics21.org/2019/11/07/funcoid-is-a-structure-in-the-sense-of-math-logic/} (blog post).

\url{https://arxiv.org/abs/1910.01014} Codensity: Isbell duality, pro-objects, compactness and accessibility.

\url{https://arxiv.org/abs/1910.12228} New proofs for some fundamental results of topology.

\url{https://arxiv.org/abs/1910.12228} A simple proof of Tychonoff theorem.

\url{https://arxiv.org/abs/1910.05293} Lusin and Suslin properties of function spaces.

\url{https://www.researchgate.net/project/Generalized-topological-groups-in-Delfs-Knebusch-generalized-topology}
Generalized topological groups in Delfs-Knebusch generalized topology.

\url{https://www.researchgate.net/publication/334759733_Closed_subsets_of_compact-like_topological_spaces}
Closed subsets of compact-like topological spaces.

\url{ttps://arxiv.org/abs/1906.10832} Existence of well-filterifications of $T_0$ topological spaces.

\url{https://arxiv.org/abs/1906.11194} Locally ordered topological spaces.

\url{https://arxiv.org/abs/1906.03549} Supercompact minus compact is super.

\url{https://arxiv.org/abs/1905.12446} On $\mathcal{H}_Y$-Ideals.

\url{https://arxiv.org/abs/1910.12228} New proofs for some fundamental results of topology.

\url{https://arxiv.org/abs/1906.08498} A $T_0$ Compactification Of A Tychonoff Space Using The Rings Of Baire One
  Functions.

\url{https://arxiv.org/abs/1911.05390} Soft $T_{(0,\alpha)}$ Spaces.

Regarding my diagrams where every loop is identity:
A category in which every endomorphism is an identity is called a one-way category.

\section{Common generalizations of funcoids and convergence spaces}

I propose the following (possible) common generalizations of funcoids and convergence spaces~(\cite{converg}):

``What you call "prefunctors" are more commonly known as semifunctors.''

https://math.stackexchange.com/q/154336/4876

\url{https://en.wikipedia.org/w/index.php?title=Equicontinuity&oldid=1155421364} distinguishes
pointwise equicontinuity and uniform equicontinuity. Also note what that article tells about compact spaces.
Also that article talks about ``evenly continuous''.

\url{https://en.wikipedia.org/wiki/Topological_vector_space}

\url{https://arxiv.org/pdf/2306.07977.pdf} Primal proximity spaces.

\begin{itemize}
\item To every set we associate an isotone (and in some sense preserving finite joins) collection of filters.
\item To every filter we associate an isotone (and in some sense preserving finite joins) collection of filters.
\item Consider pointfree funcoids between isotone families of filters.
\item What's about ``double-filtrator'' $(A;B;C)$?
\end{itemize}

\section{Star-morphisms}

Generalize ordered semigroup actions as star-categories or more generally:

Trans-precategory is a set with composition parametrized by bijection of subsets of indexes, where the result is on exclusive join of remaining indexes.

Special case (also about star-categories) composition over a pair of indexes.

Consider what is a special case of what.

\section{Dimension}

Define dimensions of funcoids.

Every funcoid of dimension~$N$ can be represented as
a subfuncoid of a composition $f_n\circ\dots\circ f_0$ of $N$-planes
$f_0$, \dots, $f_n$? Seems wrong, counterexample:
$\bigcup_{i\in\mathbb{N}} l_{\frac{1}{i+1}}$ where $l_\alpha$ is the
abscissa rotated~$\alpha$ radians.

Don't forget to add open problems to
\url{http://openproblemgarden.org} and
\url{https://scilag.net}.

\bibliographystyle{plain}
\bibliography{refs}

\end{document}
