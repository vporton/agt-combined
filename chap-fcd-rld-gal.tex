\chapter{Extending Galois connections between funcoids and reloids}

\begin{defn}
~
\begin{enumerate}
\item $\Phi_{\ast} f = \lambda b \in \mathfrak{B}: \bigsqcup \setcond{ x \in
\mathfrak{A} }{ f x \sqsubseteq b }$;
\item $\Phi^{\ast} f = \lambda b \in \mathfrak{A}: \bigsqcap \setcond{ x \in
\mathfrak{B} }{ f x \sqsupseteq b }$.
\end{enumerate}
\end{defn}

\begin{prop}
  ~
  \begin{enumerate}
    \item If $f$ has upper adjoint then $\Phi_{\ast} f$ is the upper adjoint
    of $f$.
    
    \item If $f$ has lower adjoint then $\Phi^{\ast} f$ is the lower adjoint
    of $f$.
  \end{enumerate}
\end{prop}

\begin{proof}
  By theorem \bookref{adj-max}.
\end{proof}

\begin{lem}
  $\Phi^{\ast} \torldout = \tofcd$.
\end{lem}

\begin{proof}
  $(\Phi^{\ast} \torldout) f = \bigsqcap \setcond{ g
  \in \mathsf{FCD} }{
 \torldout g \sqsupseteq f } =
  \bigsqcap^{\mathsf{FCD}} \setcond{ g \in \mathbf{Rel}
  }{ \torldout g \sqsupseteq
  f } = \bigsqcap^{\mathsf{FCD}} \setcond{ g \in
  \mathbf{Rel} }{ g \sqsupseteq f } =
  \tofcd f$.
\end{proof}

\begin{lem}
  $\Phi_{\ast} \torldout \neq
  \tofcd$.
\end{lem}

\begin{proof}
  $(\Phi_{\ast} \torldout) f = \bigsqcup \setcond{ g
  \in \mathsf{FCD} }{
 \torldout g \sqsubseteq f }$
  
  $(\Phi_{\ast} \torldout) \bot \neq \bot$.
\end{proof}

\begin{lem}
  $\Phi^{\ast}  \tofcd = \torldout$.
\end{lem}

\begin{proof}
  $(\Phi^{\ast}  \tofcd) f = \bigsqcap \setcond{ g \in
  \mathsf{RLD} }{ \tofcd g
  \sqsupseteq f } = \bigsqcap^{\mathsf{RLD}} \setcond{ g \in \mathbf{Rel}
  }{ \tofcd g \sqsupseteq f } =
  \bigsqcap^{\mathsf{RLD}} \setcond{ g \in \mathbf{Rel} }{ g
  \sqsupseteq f } = \torldout f$.
\end{proof}

\begin{lem}
  $\Phi_{\ast} \torldin = \tofcd$.
\end{lem}

\begin{proof}
  $(\Phi_{\ast} \torldin) f = \bigsqcup \setcond{ g
  \in \mathsf{FCD} }{
 \torldin g \sqsubseteq f } = \bigsqcup
  \setcond{ g \in \mathsf{FCD} }{ g \sqsubseteq
  \tofcd f } = \tofcd f$.
\end{proof}

\begin{thm}
The picture at figure~\ref{dia:fcd-rld-gal} describes values of functions~$\Phi_{\ast}$ and~$\Phi^{\ast}$.
All nodes of this diagram are distinct.
\begin{figure}
  \begin{tikzcd}
    %\arrow[loop left]{l}{\Phi_{\ast}} \id \arrow[loop right]{r}{\Phi^{\ast}} \\
    \tofcd %\arrow{u}{\Phi^{\ast}}
      \arrow[leftrightarrow, shift left=1.0ex]{r}{\Phi_{\ast}} &
      \torldin \arrow[shift left=1.0ex]{l}{\Phi^{\ast}} \\
    \torldout \arrow[leftrightarrow]{u}{\Phi^{\ast}} \arrow{d}{\Phi_{\ast}} \\
    \text{other}
  \end{tikzcd}
  \caption{\label{dia:fcd-rld-gal}}
\end{figure}
\end{thm}

\begin{proof}
Follows from the above lemmas.
\end{proof}

\begin{question}
What is at the node ``other''?
\end{question}

Trying to answer this question:

\begin{lem}
$(\Phi_{\ast}\torldout)\bot = \Omega^{\mathsf{FCD}}$.
\end{lem}

\begin{proof}
We have $\torldout\Omega^{\mathsf{FCD}} = \bot$.
$x\nsqsubseteq\Omega^{\mathsf{FCD}} \Rightarrow
\torldout x\sqsupseteq\Cor x\sqsupset\bot$.
Thus $\max\setcond{x\in\mathsf{FCD}}{\torldout x=\bot}=
\Omega^{\mathsf{FCD}}$.

So $(\Phi_{\ast}\torldout)\bot = \Omega^{\mathsf{FCD}}$.
\end{proof}

\begin{conjecture}
$(\Phi_{\ast}\torldout)f =
\Omega^{\mathsf{FCD}}\sqcup\tofcd f$.
\end{conjecture}

The above conjecture looks not natural, but I do not see
a better alternative formula.

\begin{question}
What happens if we keep applying $\Phi^{\ast}$ and $\Phi_{\ast}$ to the node ``other''?
Will we this way get a finite or infinite set?
\end{question}
