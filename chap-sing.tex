\chapter{Singularities}

\textbf{Very} rough draft.

\section{Singularities funcoids: some special cases}
We attempt to prove that $\up z$ is closed regarding finite intersections.

For consideration of this, let's consider two special cases (first of which is a specialization of the second).

Let $\mu=\nu$ be the natural proximity on real numbers $\mathbb{R}$.

Let $\Delta$ is the entourage filter of zero.

1. $z=\Delta\times^{\mathsf{FCD}}\Delta$.

2. $z=\nu\circ (\uparrow^{\mathsf{FCD}} f)|_{\Delta}$ for an arbitrary function $f:\mathbb{R}\rightarrow\mathbb{R}$.

(1) is [[also formulated in elementary terms|http://math.stackexchange.com/questions/568513/is-a-set-closed-under-finite-intersections-about-filters]] (without using funcoids).

These two above conjectures are shown to be false by a counter-example in [[this blog post|http://portonmath.wordpress.com/2013/12/18/a-negative-result-on-a-conjecture/]]. It is a discouraging result as it seems from it the plain funcoids can't be used for the multilevel theory of singularities.

\section{Using plain funcoids}
This way if we succeed is the best way to create metasingular numbers because, it (if we succeed) involves just funcoids not some fancy generalization of funcoids.

Approximate definition of "singularity level": //Singularity level// is a transitive, $T_2$-separable endofuncoid.

Now define the funcoid $\nu_{i+1}=\operatorname{SLA}(\nu_i)$:

$\operatorname{Ob}(\nu_{i+1})$ is defined as the set of all generalized limits (having fixed $\mu$, $\nu$, and $G$).

$X \mathrel{[ \nu_{i+1}]^{\ast}} Y \Leftrightarrow \exists z \in \bigcup \operatorname{Ob} \nu \forall K \in \operatorname{up} z \exists x \in \bigcup X, y \in \bigcup Y : x, y\sqsubseteq K$.

The trouble is to prove that the funcoid $\nu_{i+1}$ exists (is really a funcoid).

$\neg(X \mathrel{[ \nu_{i+1}]^{\ast}} \emptyset)$ and $\neg(\emptyset \mathrel{[ \nu_{i+1}]^{\ast}} Y)$ are obvious. We need to prove
$$I\cup J \mathrel{[ \nu_{i+1}]^{\ast}} Y \Leftrightarrow I \mathrel{[ \nu_{i+1}]^{\ast}} Y \vee J \mathrel{[ \nu_{i+1}]^{\ast}} Y$$ and
$$X \mathrel{[ \nu_{i+1}]^{\ast}} I\cup J \Leftrightarrow X \mathrel{[ \nu_{i+1}]^{\ast}} I \vee X \mathrel{[ \nu_{i+1}]^{\ast}} J.$$

Let's attempt to prove the first of the above equations (the second is dual).

$I \cup J \mathrel{[ \operatorname{SLA} ( \nu)]^{\ast}} Y \Leftrightarrow \\ \exists z
\in \bigcup \operatorname{Ob} \nu \forall K \in \operatorname{up} z \exists x \in \bigcup I \cup \bigcup J, y \in \bigcup Y :
x, y \sqsubseteq K \Leftrightarrow \\
\exists z \in \bigcup \operatorname{Ob} \nu \forall K \in \operatorname{up} z : ( \exists x \in \bigcup I \cup
\bigcup J : x \sqsubseteq K \wedge \exists y \in \bigcup Y : y \sqsubseteq K) \Leftrightarrow \\
\exists z \in \bigcup \operatorname{Ob} \nu \forall K \in \operatorname{up} z \exists x \in \bigcup I \cup \bigcup J :
x \sqsubseteq K \wedge \\ \exists z \in \bigcup \operatorname{Ob} \nu \forall K \in \operatorname{up} z
\exists y \in \bigcup Y : y \sqsubseteq K \Leftrightarrow \\
?? \\
\exists z \in \bigcup \operatorname{Ob} \nu : ( \forall K \in \operatorname{up} z \exists x \in \bigcup I : x
\sqsubseteq K \vee \\ \forall K \in \operatorname{up} z \exists x \in \bigcup J : x \sqsubseteq
K) \wedge \exists z \in \bigcup \operatorname{Ob} \nu \forall K \in \operatorname{up} z \exists y \in
\bigcup Y : y \sqsubseteq K \Leftrightarrow \\
( \exists z \in \bigcup \operatorname{Ob} \nu \forall K \in \operatorname{up} z \exists x \in \bigcup I : x
\sqsubseteq K \vee \\ \exists z \in \bigcup \operatorname{Ob} \nu \forall K \in \operatorname{up} z
\exists x \in \bigcup J : x \sqsubseteq K) \wedge \exists z \in \bigcup \operatorname{Ob} \nu \forall
K \in \operatorname{up} z \exists y \in \bigcup Y : y \sqsubseteq K \Leftrightarrow \\
( \exists z \in \bigcup \operatorname{Ob} \nu \forall K \in \operatorname{up} z \exists x \in \bigcup I : x
\sqsubseteq K \wedge \exists z \in \bigcup \operatorname{Ob} \nu \forall K \in \operatorname{up} z
\exists y \in \bigcup Y : y \sqsubseteq K) \vee \\ ( \exists z \in \bigcup \operatorname{Ob} \nu \forall
K \in \operatorname{up} z \exists x \in \bigcup J : x \sqsubseteq K \wedge \exists z \in
\bigcup \operatorname{Ob} \nu \forall K \in \operatorname{up} z \exists y \in \bigcup Y : y \sqsubseteq K)
\Leftrightarrow \\
( \exists z \in \bigcup \operatorname{Ob} \nu : ( \forall K \in \operatorname{up} z \exists x \in \bigcup I :
x \sqsubseteq K \wedge \forall K \in \operatorname{up} z \exists y \in \bigcup Y : y
\sqsubseteq K)) \vee \\ ( \exists z \in \bigcup \operatorname{Ob} \nu : ( \forall K \in \operatorname{up}
z \exists x \in \bigcup J : x \sqsubseteq K \wedge \forall K \in \operatorname{up} z \exists y
\in \bigcup Y : y \sqsubseteq K)) \Leftrightarrow \\
( \exists z \in \bigcup \operatorname{Ob} \nu : ( \forall K \in \operatorname{up} z : ( \exists x \in
\bigcup I : x \sqsubseteq K \wedge \exists y \in \bigcup Y : y \sqsubseteq K))) \vee \\ \exists z
\in \bigcup \operatorname{Ob} \nu \forall K \in \operatorname{up} z : ( \exists x \in \bigcup J : x
\sqsubseteq K \wedge \exists y \in \bigcup Y : y \sqsubseteq K) \Leftrightarrow \\
I \mathrel{[ \operatorname{SLA} ( \nu)]^{\ast}} Y \vee J \mathrel{[ \operatorname{SLA} (
\nu)]^{\ast}} Y$.

To finish the proof we need to fulfill ?? in the above formula. For this it's enough to prove

$\forall K \in \operatorname{up} z \exists x \in \bigcup I\cup \bigcup J : x \sqsubseteq K \Rightarrow \\ \forall K \in \operatorname{up}
z \exists x \in \bigcup I : x \sqsubseteq K \vee \forall K \in \operatorname{up} z \exists x
\in \bigcup J : x \sqsubseteq K$.

If $z=\uparrow Z$ is a principal funcoid, then

$\forall K \in \operatorname{up} z \exists x \in \bigcup I\cup \bigcup J : x \sqsubseteq K \Rightarrow \\ 
\exists x \in \bigcup I\cup \bigcup J : x \sqsubseteq z \Rightarrow \\
\exists x \in \bigcup I : x \sqsubseteq z \vee \exists x \in \bigcup J : x \sqsubseteq z \Rightarrow \\
\forall K \in \operatorname{up}
z \exists x \in \bigcup I : x \sqsubseteq K \vee \forall K \in \operatorname{up} z \exists x
\in \bigcup J : x \sqsubseteq K$.

Following the idea of [[the proof in this math.stackexchange.com question|http://math.stackexchange.com/questions/562908/an-implication-involving-filters\#562974]] it is easy to show that our implication is true if $\operatorname{up} z$ is closed regarding finite meets. See [[this page|Singularities funcoids: some special cases]] for attempts to set it true.
The question is whether our statement holds for non-principal funcoids. Or is there a counterexampe?

\section{Singularities funcoids: special cases proof attempts}
To prove that $\operatorname{GR} ( \Delta \times^{\mathsf{FCD}} \Delta)$ is closed under finite intersections, it's enough to prove that for every $f \in \operatorname{GR} ( \Delta \times^{\mathsf{FCD}} \Delta)$ there is a positive $\varepsilon$ such that $\forall x \in ] - \varepsilon ; \varepsilon[ : f x \in \Delta$.

Really, under this assumption:

For $g \in \operatorname{GR} ( \Delta \times^{\mathsf{FCD}} \Delta)$ exists $\zeta > 0$ such that $\forall x \in ] - \zeta ; \zeta[ : g x \in \Delta$. Let $\eta = \min \{ \varepsilon, \zeta \}$. So $\forall x \in ] - \eta ; \eta[ : ( \langle f \rangle x \in \Delta \wedge \langle g \rangle x \in \Delta)$ and so $\forall x \in ] - \eta ; \eta[ : \langle f \cap g \rangle x \in \Delta$ that is $\forall x \in ] - \eta ; \eta[ : \langle \uparrow^{\mathsf{FCD}} ( f \cap g) \rangle^{\ast} \{ x \} \sqsupseteq \Delta$ and consequently $f \cap g \in \operatorname{GR} ( \Delta \times^{\mathsf{FCD}} \Delta)$.
