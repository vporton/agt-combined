
\chapter{Postface}

See this Web page for my research plans: \href{http://www.mathematics21.org/agt-plans.html}{http://www.mathematics21.org/agt-plans.html}

I deem that now the most important research topics in Algebraic General
Topology are:
\begin{itemize}
\item to solve the open problems mentioned in this work;
\item research pointfree reloids (see below);
\item define and research compactness of funcoids;
\item research categories related with funcoids and reloids;
\item research multifuncoids and staroids in more details;
\item research generalized limit of compositions of functions;
\item research more on complete pointfree funcoids.
\end{itemize}
All my research of funcoids and reloids is presented at

\href{http://www.mathematics21.org/algebraic-general-topology.html}{http://www.mathematics21.org/algebraic-general-topology.html}

Please write to \href{mailto:porton@narod.ru}{porton@narod.ru}, if
you discover anything new related with my theory.


\section{Pointfree reloids}

Let us define something (let call it \emph{pointfree reloids}) corresponding
to pointfree funcoids in the same way as reloids correspond to funcoids.

First note that $\mathsf{RLD}(A,B)$ are isomorphic to $\mathfrak{F}\subsets(\subsets X\times\subsets B)$.
Then note that $\subsets(\subsets A\times\subsets B)$ are isomorphic
both to $\mathsf{pFCD}(\subsets A,\subsets B)$ and to $\atoms^{\subsets A}\times\atoms^{\subsets B}$.

But $\mathsf{FCD}(A,B)$ is isomorphic to $\mathsf{pFCD}(\mathfrak{F}(A),\mathfrak{F}(B))$.

Thus both $\mathfrak{F}\mathsf{pFCD}(\mathfrak{A},\mathfrak{B})$
and $\mathfrak{F}(\atoms^{\mathfrak{A}}\times\atoms^{\mathfrak{B}})$
correspond to $\mathsf{pFCD}(\mathfrak{F}(\mathfrak{A}),\mathfrak{F}(\mathfrak{B}))$
in the same way (replace $\subsets A\rightarrow\mathfrak{A}$, $\subsets B\rightarrow\mathfrak{B}$)
as $\mathsf{RLD}(A,B)$ corresponds to $\mathsf{FCD}(A,B)$.

So we can name either $\mathfrak{F}\mathsf{pFCD}(\mathfrak{A},\mathfrak{B})$
or $\mathfrak{F}(\atoms^{\mathfrak{A}}\times\atoms^{\mathfrak{B}})$
as \emph{pointfree reloids}.

Yes another possible way is to define pointfree reloids as the set of filters on the poset of Galois connections
between two posets.

Note that there are three different definitions of pointfree reloids.
They probably are not the same for arbitrary posets~$\mathfrak{A}$
and~$\mathfrak{B}$.

I have defined pointfree reloids, but have not yet started to research
their properties.

Research convergence for pointfree funcoids (should be easy).


\section{Formalizing this theory}

Despite of all measures taken, it is possible that there are errors
in this book. While special cases, such as filters of powersets or
funcoids, are most likely correct, general cases (such as filters
on posets or pointfree funcoids) may possibly contain wrong theorem
conditions.

Thus it would be good to formalize the theory presented in this book
in a proof assistant\footnote{A \emph{proof assistant} is a computer program which checks mathematical
proofs written in a formal language understandable by computer.} such as Coq.

If you want to work on formalizing this theory, please let me know.

See also \href{https://coq.inria.fr/bugs/show_bug.cgi?id=2957}{https://coq.inria.fr/bugs/show\_{}bug.cgi?id=2957}
