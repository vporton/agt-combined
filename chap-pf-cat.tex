\chapter{Categories related with funcoids}

I consider some categories related with pointfree funcoids.

\section{Draft status}

This is a rough partial draft.

\section{Topic of this article}

In this article are considered some categories related to \emph{pointfree
funcoids}.

\section{Category of continuous morphisms}

I will denote $\Ob f$ the object (source and destination) of an
endomorphism $f$.

\begin{defn}
  Let $C$ is a partially ordered category. The category
  $\cont (C)$ (which I call \emph{the category of
  continuous morphism} over $C$) is:
  \begin{itemize}
    \item Objects are endomorphisms of category $C$.
    
    \item Morphisms are triples $(f , a , b)$ where $a$ and $b$ are objects
    and $f : \Ob a \rightarrow \Ob b$ is a morphism of the
    category $C$ such that $f \circ a \sqsubseteq b \circ f$.
    
    \item Composition of morphisms is defined by the formula $(g , b , c)
    \circ (f , a , b) = (g \circ f , a , c)$.
    
    \item Identity morphisms are $(a , a , 1^C_a)$.
  \end{itemize}
\end{defn}

It is really a category:

\begin{proof}
  We need to prove that: composition of morphisms is a morphism, composition
  is associative, and identity morphisms can be canceled on the left and on
  the right.
  
  That composition of morphisms is a morphism follows from these implications:
  \[ f \circ a \sqsubseteq b \circ f \wedge g \circ b \sqsubseteq c \circ g
     \Rightarrow g \circ f \circ a \sqsubseteq g \circ b \circ f \sqsubseteq c
     \circ g \circ f. \]
  That composition is associative is obvious.
  
  That identity morphisms can be canceled on the left and on the right is
  obvious.
\end{proof}

\begin{rem}
  The ``physical'' meaning of this category is:
  \begin{itemize}
    \item Objects (endomorphisms of $C$) are spaces.
    
    \item Morphisms are continuous functions between spaces.
    
    \item $f \circ a \sqsubseteq b \circ f$ intuitively means that $f$
    combined with an infinitely small is less than infinitely small combined
    with $f$ (that is $f$ is continuous).
  \end{itemize}
\end{rem}

\begin{rem}
  Every $\Hom (\mathfrak{A}, \mathfrak{B})$ of $\mathbf{Pos}$
  is partially ordered by the formula $a \leqslant b \Leftrightarrow \forall x
  \in \mathfrak{A}: a (x) \leqslant b (x)$. So $\cont
  (\mathbf{Pos})$ is defined.
\end{rem}

\begin{defn}
  I call a $\mathbf{Pos}$-morphism \emph{monovalued} when it maps
  atoms to atoms or least element.
\end{defn}

\begin{defn}
  I call a $\mathbf{Pos}$-morphism \emph{entirely defined} when
  its value is non-least on every non-least element.
\end{defn}

\begin{obvious}
A morphism is both monovalued and entirely defined iff it maps atoms into
atoms.
\end{obvious}

\fxnote{Show how it relates with dagger categories.}

\begin{defn}
  $\mathbf{mePos}$ is the subcategory of $\mathbf{Pos}$ with
  only monovalued and entirely defined morphisms.
\end{defn}

\begin{obvious}
This is a well defined category.{\hspace*{\fill}}{\medskip}
\end{obvious}

\begin{defn}
  $\mathbf{mefp} \mathsf{FCD}$ is the subcategory of
  $\mathbf{fp} \mathsf{FCD}$ with only monovalued and entirely
  defined morphisms.
\end{defn}

\begin{rem}
  In the two above definitions different definitions of monovaluedness and
  entire definedness from different articles.
\end{rem}

\section{Definition of the categories}

\begin{defn}
  A \emph{(pointfree) endo-funcoid} is a (pointfree) funcoid with the same
  source and destination (an endomorphism of the category of (pointfree)
  funcoids). I will denote $\Ob f$ the object of an endomorphism $f$.
\end{defn}

\begin{obvious}
The \emph{category of continuous pointfree funcoids} $\cont
(\mathbf{fp} \mathsf{FCD})$ is:
\begin{itemize}
  \item Objects are small pointfree endo-funcoids.
  
  \item Morphisms from an object $a$ to an object $b$ are triples $(f , a ,
  b)$ where $f$ is a pointfree funcoid from $\Ob a$ to $\Ob b$
  such that $f$ is a continuous morphism from $a$ to $b$ (that is $f \circ a
  \sqsubseteq b \circ f$, or equivalently $a \sqsubseteq f^{- 1} \circ b \circ
  f$, or equivalently $f \circ a \circ f^{- 1} \sqsubseteq f$).
  
  \item Composition is the composition of pointfree funcoids.
  
  \item Identity for an object $a$ is $(I^{\mathsf{FCD}}_{\Ob a}
  , a , a)$.
\end{itemize}
\end{obvious}

\section{Isomorphisms}

\begin{thm}
  If $f$ is an isomorphism $a \rightarrow b$ of the category
  $\cont (\mathbf{fp}
  \mathsf{FCD})$, then:
  \begin{enumerate}
    \item $f \circ a = b \circ f$;
    
    \item $a = f^{- 1} \circ b \circ f$;
    
    \item $f \circ a \circ f^{- 1} = b$.
  \end{enumerate}
\end{thm}

\begin{proof}
  Note that $f$ is monovalued and entirely defined.
  
  1. We have $f \circ a \sqsubseteq b \circ f$ and $f^{- 1} \circ b
  \sqsubseteq a \circ f^{- 1}$. Consequently $f^{- 1} \circ f \circ a
  \sqsubseteq f^{- 1} \circ b \circ f$; $a \sqsubseteq f^{- 1} \circ b \circ
  f$; $a \circ f^{- 1} \sqsubseteq f^{- 1} \circ b \circ f \circ f^{- 1}$; $a
  \circ f^{- 1} \sqsubseteq f^{- 1} \circ b$. Similarly $b \circ f \sqsubseteq
  f \circ a$. So $f \circ a = b \circ f$.
  
  2 and 3. Follow from the definition of isomorphism.
\end{proof}

Isomorphisms are meant to preserve structure of objects. I will show that
(under certain conditions) isomorphisms of $\cont
(\mathbf{fp} \mathsf{FCD})$ really preserve
structure of objects.

First we will consider an isomorphism between objects $a$ and $b$ which are
funcoids (not the general case of pointfree funcoids). In this case a map
which preserves structure of objects is a \emph{bijection}. It is really a
bijection as the following theorem says:

\begin{thm}
If $f$ is an isomorphism of the category of funcoids then $f$ is a discrete
funcoid (so, it is essentially a bijection).
\fxnote{Split it into two propositions: about completeness and co-completeness.}
\end{thm}

\begin{proof}
  $\supfun{f}^{\ast} A \sqcap \supfun{f}^{\ast} ((\Src f)
  \setminus A) = 0^{\Dst f}$ because $f$ is monovalued.
  
  $\supfun{f}^{\ast} A \sqcup \supfun{f}^{\ast} ((\Src f)
  \setminus A) = 1^{\Dst f}$.
  
  Therefore $\supfun{f}^{\ast} A$ is a principal filter (theorem 49 in
  {\cite{filters}}). So $f$ is co-complete.
  
  That $f$ is complete follows from symmetry.
\end{proof}

For wider class of pointfree funcoids the concept of bijection does not make
sense. Instead we would want a structure preserving map to be \emph{order
isomorphism}.

Actually, for mapping between $\subsets A$ and $\subsets B$ where $A$
and $B$ are some sets (including the above considered case of funcoids from
$A$ to $B$) bijection and order isomorphism are essentially the same:

\begin{prop}
  Bijections $F$ between sets $A$ and $B$ bijectively correspond to order
  isomorphisms $f$ between $\subsets A$ and $\subsets B$ by the formula
  $f = \supfun{F}$.
\end{prop}

\begin{proof}
  Let $F$ is a bijection. Then $X \subseteq Y \Rightarrow \supfun{F} X
  \subseteq \supfun{F} Y$ and $\langle F^{- 1} \rangle \langle F
  \rangle X = X$ for every sets $X, Y \in \subsets A$. Thus $f = \langle F
  \rangle$ is an order isomorphism.
  
  Let now $f$ is an order isomorphism between $\subsets A$ and $\subsets
  B$. Then $f (\{ x \})$ is a singleton for every $x \in A$. Take $F (x)$ to
  the unique $y$ such that $f (\{ x \}) = \{ y \}$. Obviously $f$ is a
  bijection and $f = \supfun{F}$.
\end{proof}

For arbitrary pointfree funcoids isomorphisms do not necessarily preserve
structure. It holds only for \emph{increasing pointfree funcoids}:

\begin{defn}
  I call a pointfree funcoid $f$ \emph{increasing} iff $\supfun{f}$
  and $\langle f^{- 1} \rangle$ are monotone functions.
\end{defn}

\begin{prop}
  If $f$ is an increasing isomorphism of the category of pointfree funcoids
  then $\supfun{f}$ is an order isomorphism.
\end{prop}

\begin{proof}
  We have: $\supfun{f} \circ \langle f^{- 1} \rangle = \langle f \circ
  f^{- 1} \rangle = \langle \id^{\mathsf{FCD}}_{\mathfrak{B}}
  \rangle = \id_{\mathfrak{B}}$ and $\langle f^{- 1} \rangle \circ
  \supfun{f} = \langle f^{- 1} \circ f \rangle = \langle
  \id^{\mathsf{FCD}}_{\mathfrak{A}} \rangle =
  \id_{\mathfrak{A}}$. Thus $\supfun{f}$ is a bijection.
  
  $\supfun{f}$ is increasing and bijective.
\end{proof}

\begin{rem}
  Non-increasing isomorphisms of the category of pointfree funcoids are
  against sound mind, they don't preserve the structure of the source, that is
  for them $\supfun{f}$ or $\langle f^{- 1} \rangle$ are not order
  isomorphisms.
\end{rem}

\begin{obvious}
Isomorphisms of $\cont (\mathbf{Pos})$ and
$\cont (\mathbf{mePos})$ are order
isomorphisms.
\end{obvious}

\section{Direct products}

\fxerror{Now this section is a complete mess. Clean it up.}

Consider the category $\mathbf{contFcd}$ which is the full
subcategory $\cont (\mathbf{mePos})$ restricted to
objects which are essentially increasing pointfree funcoids.

Let $f_1 : Y \rightarrow X_1$ and $f_2 : Y \rightarrow X_2$ are morphisms of
$\mathbf{contFcd}$.

The product object is $X_1 \times^{(C)} X_2$ (cross composition product of
funcoids used). It is easy to see that $X_1 \times^{(C)} X_2$ is an object of
$\mathbf{contFcd}$ that is an endo-funcoid.

The morphism $f_1 \times^{(D)} f_2 : Y \rightarrow X_1 \times^{(C)} X_2$ is
defined by the formula $(f_1 \times^{(D)} f_2) y = f_1 y
\times^{\mathsf{FCD}} f_2 y$.

$f_1 \times^{(D)} f_2$ is monovalued and entirely defined because so are $f_1$
and $f_2$.
\[ (f_1 \times^{(D 2)} f_2) y = \bigcup \left\{ f_1 Y
   \times^{\mathsf{FCD}} f_2 Y \hspace{1em} | \hspace{1em} Y \in
   \atoms^{\mathfrak{A}} y \right\} . \]

\fxnote{Is $(f_1 \times^{(D 2)} f_2)$ a pointfree funcoid?}

To prove that it is really a morphism we need to show
\[ (f_1 \times^{(D)} f_2) \circ Y \sqsubseteq (X_1 \times^{(C)} X_2) \circ
   (f_1 \times^{(D)} f_2) \]
that is (for every $y$)
\[ (f_1 \times^{(D)} f_2) Y y \sqsubseteq (X_1 \times^{(C)} X_2) (f_1
   \times^{(D)} f_2) y. \]
Really, $(f_1 \times^{(D)} f_2) Y y = f_1 Y y \times^{\mathsf{FCD}} f_2
Y y$;

$(X_1 \times^{(C)} X_2) (f_1 \times^{(D)} f_2) y = (X_1 \times^{(C)} X_2) (f_1
y \times^{\mathsf{FCD}} f_2 y) = X_1 f_1 y \times^{\mathsf{FCD}}
X_2 f_2 y$;

but it is easy to show $f_1 Y y \times^{\mathsf{FCD}} f_2 Y y
\sqsubseteq X_1 f_1 y \times^{\mathsf{FCD}} X_2 f_2 y$.

??

I define ??

\fxnote{Prove that it is a direct product in $\mathbf{contFcd}$.}